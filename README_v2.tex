\documentclass[ignorenonframetext,]{beamer}
\setbeamertemplate{caption}[numbered]
\setbeamertemplate{caption label separator}{: }
\setbeamercolor{caption name}{fg=normal text.fg}
\beamertemplatenavigationsymbolsempty
\usepackage{lmodern}
\usepackage{amssymb,amsmath}
\usepackage{ifxetex,ifluatex}
\usepackage{fixltx2e} % provides \textsubscript
\ifnum 0\ifxetex 1\fi\ifluatex 1\fi=0 % if pdftex
  \usepackage[T1]{fontenc}
  \usepackage[utf8]{inputenc}
\else % if luatex or xelatex
  \ifxetex
    \usepackage{mathspec}
  \else
    \usepackage{fontspec}
  \fi
  \defaultfontfeatures{Ligatures=TeX,Scale=MatchLowercase}
\fi
% use upquote if available, for straight quotes in verbatim environments
\IfFileExists{upquote.sty}{\usepackage{upquote}}{}
% use microtype if available
\IfFileExists{microtype.sty}{%
\usepackage{microtype}
\UseMicrotypeSet[protrusion]{basicmath} % disable protrusion for tt fonts
}{}
\newif\ifbibliography
\hypersetup{
            pdfborder={0 0 0},
            breaklinks=true}
\urlstyle{same}  % don't use monospace font for urls
\usepackage{color}
\usepackage{fancyvrb}
\newcommand{\VerbBar}{|}
\newcommand{\VERB}{\Verb[commandchars=\\\{\}]}
\DefineVerbatimEnvironment{Highlighting}{Verbatim}{commandchars=\\\{\}}
% Add ',fontsize=\small' for more characters per line
\usepackage{framed}
\definecolor{shadecolor}{RGB}{248,248,248}
\newenvironment{Shaded}{\begin{snugshade}}{\end{snugshade}}
\newcommand{\KeywordTok}[1]{\textcolor[rgb]{0.13,0.29,0.53}{\textbf{#1}}}
\newcommand{\DataTypeTok}[1]{\textcolor[rgb]{0.13,0.29,0.53}{#1}}
\newcommand{\DecValTok}[1]{\textcolor[rgb]{0.00,0.00,0.81}{#1}}
\newcommand{\BaseNTok}[1]{\textcolor[rgb]{0.00,0.00,0.81}{#1}}
\newcommand{\FloatTok}[1]{\textcolor[rgb]{0.00,0.00,0.81}{#1}}
\newcommand{\ConstantTok}[1]{\textcolor[rgb]{0.00,0.00,0.00}{#1}}
\newcommand{\CharTok}[1]{\textcolor[rgb]{0.31,0.60,0.02}{#1}}
\newcommand{\SpecialCharTok}[1]{\textcolor[rgb]{0.00,0.00,0.00}{#1}}
\newcommand{\StringTok}[1]{\textcolor[rgb]{0.31,0.60,0.02}{#1}}
\newcommand{\VerbatimStringTok}[1]{\textcolor[rgb]{0.31,0.60,0.02}{#1}}
\newcommand{\SpecialStringTok}[1]{\textcolor[rgb]{0.31,0.60,0.02}{#1}}
\newcommand{\ImportTok}[1]{#1}
\newcommand{\CommentTok}[1]{\textcolor[rgb]{0.56,0.35,0.01}{\textit{#1}}}
\newcommand{\DocumentationTok}[1]{\textcolor[rgb]{0.56,0.35,0.01}{\textbf{\textit{#1}}}}
\newcommand{\AnnotationTok}[1]{\textcolor[rgb]{0.56,0.35,0.01}{\textbf{\textit{#1}}}}
\newcommand{\CommentVarTok}[1]{\textcolor[rgb]{0.56,0.35,0.01}{\textbf{\textit{#1}}}}
\newcommand{\OtherTok}[1]{\textcolor[rgb]{0.56,0.35,0.01}{#1}}
\newcommand{\FunctionTok}[1]{\textcolor[rgb]{0.00,0.00,0.00}{#1}}
\newcommand{\VariableTok}[1]{\textcolor[rgb]{0.00,0.00,0.00}{#1}}
\newcommand{\ControlFlowTok}[1]{\textcolor[rgb]{0.13,0.29,0.53}{\textbf{#1}}}
\newcommand{\OperatorTok}[1]{\textcolor[rgb]{0.81,0.36,0.00}{\textbf{#1}}}
\newcommand{\BuiltInTok}[1]{#1}
\newcommand{\ExtensionTok}[1]{#1}
\newcommand{\PreprocessorTok}[1]{\textcolor[rgb]{0.56,0.35,0.01}{\textit{#1}}}
\newcommand{\AttributeTok}[1]{\textcolor[rgb]{0.77,0.63,0.00}{#1}}
\newcommand{\RegionMarkerTok}[1]{#1}
\newcommand{\InformationTok}[1]{\textcolor[rgb]{0.56,0.35,0.01}{\textbf{\textit{#1}}}}
\newcommand{\WarningTok}[1]{\textcolor[rgb]{0.56,0.35,0.01}{\textbf{\textit{#1}}}}
\newcommand{\AlertTok}[1]{\textcolor[rgb]{0.94,0.16,0.16}{#1}}
\newcommand{\ErrorTok}[1]{\textcolor[rgb]{0.64,0.00,0.00}{\textbf{#1}}}
\newcommand{\NormalTok}[1]{#1}

% Prevent slide breaks in the middle of a paragraph:
\widowpenalties 1 10000
\raggedbottom

\AtBeginPart{
  \let\insertpartnumber\relax
  \let\partname\relax
  \frame{\partpage}
}
\AtBeginSection{
  \ifbibliography
  \else
    \let\insertsectionnumber\relax
    \let\sectionname\relax
    \frame{\sectionpage}
  \fi
}
\AtBeginSubsection{
  \let\insertsubsectionnumber\relax
  \let\subsectionname\relax
  \frame{\subsectionpage}
}

\setlength{\parindent}{0pt}
\setlength{\parskip}{6pt plus 2pt minus 1pt}
\setlength{\emergencystretch}{3em}  % prevent overfull lines
\providecommand{\tightlist}{%
  \setlength{\itemsep}{0pt}\setlength{\parskip}{0pt}}
\setcounter{secnumdepth}{0}

\date{}

\begin{document}

\begin{frame}{R for Air Quality - Introduction}

\end{frame}

\begin{frame}{Q \& A }

\begin{itemize}
\tightlist
\item
  Please submit questions in the question box on the GoToWebinar Side
  Panel.
\item
  We will answer as many questions as possible during this webinar. If
  we can't get to your question during the webinar, we'll respond via
  email.
\end{itemize}

\end{frame}

\begin{frame}{Overview}

\begin{itemize}
\tightlist
\item
  About R
\item
  The bad vs.~the good
\item
  Coding in R

  \begin{itemize}
  \tightlist
  \item
    RStudio
  \item
    Scripts.R
  \item
    Functions
  \item
    Object names
  \item
    Importing data
  \item
    Calling columns in a dataframe with \$
  \item
    Storing variables, temporary objects
  \item
    Data types
  \item
    Data structures
  \item
    Missing values - NA
  \item
    Logical operators
  \item
    Packages
  \end{itemize}
\item
  Best practices

  \begin{itemize}
  \tightlist
  \item
    RMarkdown.Rmd
  \item
    R Projects vs.~working directories
  \item
    Debugging - 5 Steps (Jenny Bryan's talk)
  \end{itemize}
\end{itemize}

Garrett Grolemund and Hadley Wickham's book, ``R for Data Science'' was
instrumental in developing this webinar series.
\url{https://r4ds.had.co.nz/}

\end{frame}

\begin{frame}{About R}

\begin{itemize}
\tightlist
\item
  Statistical programming language
\item
  High-level
\item
  Versatile, robust
\item
  Packages
\item
  Useful for both data preparation and data analysis
\item
  Open source

  \begin{itemize}
  \tightlist
  \item
    Meaning: source code is freely available to the public, the public
    can contribute
  \end{itemize}
\item
  Free
\end{itemize}

\begin{block}{The bad}

\begin{itemize}
\tightlist
\item
  R has a steep learning curve
\item
  If colleagues don't know R, this can hinder collaboration
\end{itemize}

\end{block}

\begin{block}{The good}

\begin{itemize}
\tightlist
\item
  R can handle larger files
\item
  Excel is a spreadsheet tool, errors hide
\item
  Reproducability, transparency
\item
  Advanced and versatile statistics
\item
  Customizable data visualizations
\end{itemize}

\end{block}

\begin{block}{The different (but neutral)}

\begin{itemize}
\tightlist
\item
  R has less formal support, but virtually endless online user community
  forums
\end{itemize}

\end{block}

\end{frame}

\begin{frame}{RStudio}

\begin{itemize}
\tightlist
\item
  The most popular Integrated Development Environment (IDE) for R
\end{itemize}

\end{frame}

\begin{frame}{Scripts.R}

\end{frame}

\begin{frame}{Scripts.R}

\end{frame}

\begin{frame}[fragile]{Packages/Libraries}

\begin{itemize}
\tightlist
\item
  Contain numerous functions that help you complete specialized tasks
\item
  Anyone can make one
\item
  The reason R is so versatile and robust
\item
  Tidyverse
\item
  Openair
\item
  Many, many more
\end{itemize}

\begin{block}{Installation:}

\begin{Shaded}
\begin{Highlighting}[]
\CommentTok{# not run}
\KeywordTok{install.packages}\NormalTok{(}\StringTok{"tidyverse"}\NormalTok{)}
\end{Highlighting}
\end{Shaded}

\end{block}

\begin{block}{Call a package or library:}

\begin{Shaded}
\begin{Highlighting}[]
\KeywordTok{library}\NormalTok{(tidyverse)}
\end{Highlighting}
\end{Shaded}

\begin{verbatim}
## Warning: package 'tidyverse' was built under R version 3.6.3

## -- Attaching packages --------------------------------------------------------------------------------------------------------------------------------- tidyverse 1.3.0 --

## v ggplot2 3.3.0     v purrr   0.3.3
## v tibble  2.1.3     v dplyr   0.8.3
## v tidyr   1.0.0     v stringr 1.4.0
## v readr   1.3.1     v forcats 0.4.0

## Warning: package 'ggplot2' was built under R version 3.6.3

## -- Conflicts ------------------------------------------------------------------------------------------------------------------------------------ tidyverse_conflicts() --
## x dplyr::filter() masks stats::filter()
## x dplyr::lag()    masks stats::lag()
\end{verbatim}

\end{block}

\end{frame}

\begin{frame}[fragile]{You can use R as a calculator}

\begin{verbatim}
400 * 3 + 55.889
2 ^ 8
\end{verbatim}

You can create new objects with \textless{}- (which reads ``gets'')

\begin{verbatim}
objectName <- value
someText <- "Hello World"
print(someText)
\end{verbatim}

Tips: - Give objects short and meaningful names. - You can see objects
you've created in the upper righthand side of your RStudio window-or the
Environment - You can make a line into a comment by typing \# in front.
This is how you can add plain text or non-executable code. - Tip: press
alt + shift + k and see what happens

\end{frame}

\begin{frame}[fragile]{Functions}

R has many functions built in. These are referred to as Base R
functions. When you call a function, it looks like this:

\begin{verbatim}
function_name( arg1 = val1, arg2 = val2 )
\end{verbatim}

You can create a function, but that's beyond the scope of our training.
You will see many examples of functions as we move forward.

\end{frame}

\begin{frame}[fragile]{Object Names}

\begin{itemize}
\tightlist
\item
  Must start with a letter
\item
  Can contain \_ and .
\item
  Descriptive
\item
  Some people use snake\_case, others prefer camelCase.
\item
  Best practice: be consistent with naming conventions.
\end{itemize}

Example:

\begin{verbatim}
my_text <- "Hello World"
\end{verbatim}

\end{frame}

\begin{frame}[fragile]{Importing data}

\begin{itemize}
\tightlist
\item
  CSV's are the best file format
\item
  Other delimited files are fine too
\item
  excel workbook files can import but will complicate things
\end{itemize}

Syntax:

\begin{verbatim}
  oz_mar <- read_csv("march_ozone.csv")
\end{verbatim}

Now we have a dataframe stored in the object oz\_mar.

\begin{itemize}
\tightlist
\item
  Paths to these files is necessary if you are not already pointed to
  your working directory (more on this later!)
\item
  Many other arguments are available to specify things like whether
  there is a header, or if the delimiter is something other than a comma

  \begin{itemize}
  \tightlist
  \item
    \url{https://readr.tidyverse.org/reference/read_delim.html}
  \item
    \url{http://rprogramming.net/read-csv-in-r/}
  \end{itemize}
\end{itemize}

\end{frame}

\begin{frame}[fragile]{Calling columns in a dataframe with \$}

Example:

\begin{verbatim}
oz_mar$value
\end{verbatim}

This would pick out the column titled ``value'' within the oz\_mar
dataframe.

\end{frame}

\begin{frame}[fragile]{Storing variables, temporary objects}

When you run functions in R, you have two options for handling the
output: 1. Allow it to print to the console (it's not saved and you
can't use it later)

\begin{verbatim}
oz_mar$value * 0.9
\end{verbatim}

\begin{enumerate}
\def\labelenumi{\arabic{enumi}.}
\item
  Store it in an object using \textless{}- ``gets''

  oz\_control \textless{}- oz\_mar\$value * 0.9
\end{enumerate}

\end{frame}

\begin{frame}{Data types}

\begin{itemize}
\tightlist
\item
  Character ``Hello World''
\item
  Numeric (real or decimal) 3.14159
\item
  Integer 7
\item
  Logical TRUE
\end{itemize}

\end{frame}

\begin{frame}[fragile]{Data structures}

Atomic vector: the most basic data structure

\begin{verbatim}
c(1, 4, 6, 2, 3)
c("hello", "world", "its", "me", "jenny")
\end{verbatim}

List: similar to a vector, but it can store different data types

\begin{verbatim}
list(1, "hello", TRUE, 3.141)
\end{verbatim}

Matrix: the simplest 2-dimensional data structure

\begin{verbatim}
matrix(month.abb, nrow = 3, ncol = 4)
\end{verbatim}

Data frame: similar to matrices, tightly couples collections of
variables

Factors: a vector with ordered levels, xtremely helpful with categorical
data

\end{frame}

\begin{frame}{Missing values - NA}

\begin{itemize}
\tightlist
\item
  R knows what a missing value is (when it's represented with NA)
\item
  For automatic recognition of missing values, use NA
\item
  Sometimes NA will be the output when an operation doesn't make sense
\end{itemize}

\end{frame}

\begin{frame}{Logical operators}

You will need to use logical operators when sub-setting data and in
if-statements when your scripts get more complex (we will not cover
if-statements in this training) - ``not'' = ! - ``and'' for vectors
\textgreater{} length-1 = \& - ``and'' for single, scalar objects = \&\&
- ``or'' for vectors \textgreater{} length-1 = \textbar{} - ``or'' for
single, scalar objects = \textbar{}\textbar{} - exclusive ``or'' =
xor(x, y) - isNA() checks if an object is NA and returns TRUE or FALSE

\end{frame}

\begin{frame}{Best practices}

\begin{itemize}
\tightlist
\item
  RMarkdown files (.Rmd)
\item
  R Projects vs.~working directories
\item
  Debugging
\end{itemize}

\end{frame}

\end{document}
